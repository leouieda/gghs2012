% Latex source for "Rapid 3D inversion of gravity and gravity gradient data
% to test geologic hypotheses" by Leonardo Uieda and Valeria C F Barbosa

\RequirePackage{fix-cm}

%\documentclass[twocolumn,final]{svjour3}
\documentclass[twocolumn,draft]{svjour3}
%\documentclass[twocolumn,referee]{svjour3}

\smartqed  % flush right qed marks, e.g. at end of proof

\usepackage{graphicx}
\usepackage{amsmath}
\usepackage[round,sort]{natbib}

\journalname{International Association of Geodesy Symposia}



\begin{document}

\title{
    Rapid 3D inversion
    of gravity data
    to test geologic hypotheses
}
\author{Leonardo Uieda \and Val\'eria C. F. Barbosa}

%\subtitle{Do you have a subtitle?\\ If so, write it here}
%\titlerunning{Short form of title}        % if too long for running head
%\authorrunning{Short form of author list} % if too long for running head

\institute{
    L. Uieda \and V.C.F. Barbosa \at
        Observat\'orio Nacional,
        Rua General Jos\'e Cristino 77,
        20921-400 Rio de Janeiro - RJ, Brazil.
        \email{leouieda@gmail.com; valcris@on.br}
}

\date{Received:  / Accepted: }

\maketitle

\begin{abstract}
% Note: Max 250 words
Meh
\keywords{
    Gravity inversion \and
    Gravity gradiometry \and
    Modeling \and
    3D \and
    GOCE
}
\end{abstract}

% INTRODUCTION
%%%%%%%%%%%%%%%%%%%%%%%%%%%%%%%%%%%%%%%%%%%%%%%%%%%%%%%%%%%%%%%%%%%%%%%%%%%%%%%%
\section{Introduction}

\begin{sloppypar}
Forward modeling of potential fields
is a useful way generate geophysical models
that incorporate the interpreter's knowledge
about the regional and local geology.
However, this can be a tedious process,
specially when modeling in 3D
and/or trying to fit multiple data components,
e.g., in gravity gradiometry.
The task of simultaneously
constructing geologically realistic models
and supervising the data fit
is imposed on the interpreter.

These problems are partially solved
by methods of geophysical inversion,
which automatically fit the data.
Conversely,
inverse problems introduce other challenges of their own.
Most geophysical inverse problems are ill-posed
because their solutions are neither unique nor stable.
Thus, they require the introduction of prior information,
usually through regularizing functions \citep{silva_potinversion}.
Moreover, 3D inverse problems are computationally expensive.

Recent developments in potential field inversion
have proposed different regularizing functions
to transform the ill-posed problem into a well-posed one
\citep[e.g.,][]{last_kubik, li_oldenburg, zhdanov_focusing, silva_interactive,
silvadias_adaptive, martins_tv}.
In addition,
several techniques have been applied
to overcome the computational complexity,
like data compression \citep{zhdanov_compression, li_compression},
lazy evaluation of the sensitivity matrix \citep{uieda_planting},
and moving footprint and parallel computation \citep{cuma_largescale}.

We call attention
to the method called ``planting anomalous densities''
of \citet{uieda_planting}
that has been further developed by \citet{uieda_shape}.
This method is based on
the work of \citet{rene}.
It uses an iterative algorithm
to automatically grow the anomalous bodies
around user-specified prismatic elements called ``seeds'',
which have fixed density contrasts and positions.
These seeds provide a ``first estimate''
of the skeletal outlines of the presumed geologic bodies.
Then,
the inversion iteratively concentrates mass
around this ``skeleton''
in a way that both
fits the observed data
and yields compact bodies.
Therefore,
the interpreter can easily impose prior information
on the inversion through these seeds.
% PAREI AQUI
The interpreter needs only to supply a few seeds
that specify the sources' skeleton,
eliminating the exhaustive task
of specifying the complete geometry of multiple sources.

Moreover, the interpreter is liberated
from the time-consuming procedure
of yielding a reasonable fit to the data.

Due to its high computational efficiency,
the method of planting anomalous densities
can be used to quickly test geologic hypothesis
of different locations and density contrasts
for presumed sources.

To test a hypothesis,
one would choose
the locations and density contrasts of the seeds accordingly
and verify if the inversion result
is able to fit the observed data.

If it is not able,
then the hypothesis can be rejected
and a new one can be formulated and tested.

Otherwise, there is no reason to reject the hypothesis
on the basis of the geophysical data.

Thus, the method can be viewed as a an enhanced forward modeling.

The method of planting anomalous densities
can be used with both gravity and gravity gradient data.

This makes it an ideal tool
to interpret compact geologic bodies
using the new generation GOCE data.

We present  applications to synthetic and real data
that illustrate the usefulness of our method.

\end{sloppypar}

% METHODOLOGY
%%%%%%%%%%%%%%%%%%%%%%%%%%%%%%%%%%%%%%%%%%%%%%%%%%%%%%%%%%%%%%%%%%%%%%%%%%%%%%%%
\section{Methodology}

Meh

\section{Example application}


\begin{equation}
a^2+b^2=c^2
\end{equation}

% For one-column wide figures use
\begin{figure}
% Use the relevant command to insert your figure file.
% For example, with the graphicx package use
  %\includegraphics{example.eps}
% figure caption is below the figure
\caption{Please write your figure caption here}
\label{fig:1}       % Give a unique label
\end{figure}

% For two-column wide figures use
\begin{figure*}
% Use the relevant command to insert your figure file.
% For example, with the graphicx package use
  %\includegraphics[width=0.75\textwidth]{example.eps}
% figure caption is below the figure
\caption{Please write your figure caption here}
\label{fig:2}       % Give a unique label
\end{figure*}

% For tables use
\begin{table}
% table caption is above the table
\caption{Please write your table caption here}
\label{tab:1}       % Give a unique label
% For LaTeX tables use
\begin{tabular}{lll}
\hline\noalign{\smallskip}
first & second & third  \\
\noalign{\smallskip}\hline\noalign{\smallskip}
number & number & number \\
number & number & number \\
\noalign{\smallskip}\hline
\end{tabular}
\end{table}

% CONCLUSIONS
%%%%%%%%%%%%%%%%%%%%%%%%%%%%%%%%%%%%%%%%%%%%%%%%%%%%%%%%%%%%%%%%%%%%%%%%%%%%%%%%
\section{Conclusions}
\label{conclusions}

% ACKNOWLEDGEMENTS
%%%%%%%%%%%%%%%%%%%%%%%%%%%%%%%%%%%%%%%%%%%%%%%%%%%%%%%%%%%%%%%%%%%%%%%%%%%%%%%%
\begin{acknowledgements}
If you'd like to thank anyone, place your comments here
and remove the percent signs.
\end{acknowledgements}


% REFERENCES
%%%%%%%%%%%%%%%%%%%%%%%%%%%%%%%%%%%%%%%%%%%%%%%%%%%%%%%%%%%%%%%%%%%%%%%%%%%%%%%%
\begin{thebibliography}{}

\bibitem[\v{C}uma et~al., 2012]{cuma_largescale}
\v{C}uma M, Wilson G a., Zhdanov MS, Cuma M (2012)
Large-scale 3D inversion of potential field data. Geophysical Prospecting.
doi: 10.1111/j.1365-2478.2011.01052.x

\bibitem[Last and Kubik, 1983]{last_kubik}
Last BJ, Kubik K (1983)
Compact gravity inversion. Geophysics 48:713-721. doi: 10.1190/1.1441501

\bibitem[Li and Oldenburg, 1998]{li_oldenburg}
Li Y, Oldenburg DW (1998)
3-D inversion of gravity data. Geophysics 63:109-119.
doi: 10.1190/1.1444302

\bibitem[Li and Oldenburg, 2003]{li_compression}
Li Y, Oldenburg DW (2003)
Fast inversion of large-scale magnetic data using wavelet transforms and a
logarithmic barrier method. Geophysical Journal International 152:251-265.
doi: 10.1046/j.1365-246X.2003.01766.x

\bibitem[Martins et~al., 2011]{martins_tv}
Martins CM, Lima WA, Barbosa VCF, Silva JBC (2011)
Total variation regularization for depth-to-basement estimate:
Part 1 - Mathematical details and applications. Geophysics 76:I1-I12.
doi: 10.1190/1.3524286

\bibitem[Ren\'e, 1986]{rene}
Ren\'e RM (1986)
Gravity inversion using open, reject, and ``shape-of-anomaly'' fill criteria.
Geophysics 51:988-994. doi: 10.1190/1.1442157

\bibitem[Silva et~al., 2001]{silva_potinversion}
Silva JBC, Medeiros WE, Barbosa VCF (2001)
Potential-field inversion: Choosing the appropriate technique to solve a
geologic problem. Geophysics 66:511-520. doi: 10.1190/1.1444941

\bibitem[Silva and Barbosa, 2006]{silva_interactive}
Silva JBC, Barbosa VCF (2006)
Interactive gravity inversion. Geophysics 71:J1-J9. doi: 10.1190/1.2168010

\bibitem[Silva Dias et~al., 2009]{silvadias_adaptive}
Silva Dias FJS, Barbosa VCF, Silva JBC (2009)
3D gravity inversion through an adaptive-learning procedure.
Geophysics 74:I9-I21. doi: 10.1190/1.3092775

\bibitem[Portniaguine and Zhdanov, 2002]{zhdanov_compression}
Portniaguine O, Zhdanov MS (2002)
3‐D magnetic inversion with data compression and image focusing.
Geophysics 67:1532-1541. doi: 10.1190/1.1512749

\bibitem[Portniaguine and Zhdanov, 1999]{zhdanov_focusing}
Portniaguine O, Zhdanov MS (1999)
Focusing geophysical inversion images. Geophysics 64:874-887.
doi: 10.1190/1.1444596

\bibitem[Uieda and Barbosa, 2012a]{uieda_animation}
Uieda L, Barbosa VCF (2012a)
Animation of growth iterations during 3D gravity gradient inversion by planting
anomalous densities. Figshare. http://dx.doi.org/10.6084/m9.figshare.91469.
Accessed 06 December 2012.

\bibitem[Uieda and Barbosa, 2012b]{uieda_planting}
Uieda L, Barbosa VCF (2012b)
Robust 3D gravity gradient inversion by planting anomalous densities.
Geophysics 77:G55-G66. doi: 10.1190/geo2011-0388.1

\bibitem[Uieda and Barbosa, 2012c]{uieda_shape}
Uieda L, Barbosa VCF (2012c)
Use of the ``shape-of-anomaly'' data misfit in 3D inversion by planting
anomalous densities. SEG Technical Program Expanded Abstracts 2012.
doi: 10.1190/segam2012-0383.1

\end{thebibliography}

\end{document}

